\section{les bases}
\subsection{changer le \emph{shell} de l'utilisateur}
Ce n'est pas très compliqué, mais il faut le savoir. Il faut utiliser la commande \commande{usermod} :

\begin{lstlisting}[language=sh, frame=single, caption=changer le shell par défaut]
usermod -s /bin/zsh bernard
\end{lstlisting}

\section{les problèmes}
Oui, il y a des instabilités, par exemple avec \commande{texworks} qui plante au moment d'afficher le \LOGO{Pdf} sur un \commande{segmentation fault}. Peut-être que le mélange de deux sources de paquets y est pour quelque chose.

